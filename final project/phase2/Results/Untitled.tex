% Options for packages loaded elsewhere
\PassOptionsToPackage{unicode}{hyperref}
\PassOptionsToPackage{hyphens}{url}
%
\documentclass[
]{article}
\usepackage{amsmath,amssymb}
\usepackage{lmodern}
\usepackage{iftex}
\ifPDFTeX
  \usepackage[T1]{fontenc}
  \usepackage[utf8]{inputenc}
  \usepackage{textcomp} % provide euro and other symbols
\else % if luatex or xetex
  \usepackage{unicode-math}
  \defaultfontfeatures{Scale=MatchLowercase}
  \defaultfontfeatures[\rmfamily]{Ligatures=TeX,Scale=1}
\fi
% Use upquote if available, for straight quotes in verbatim environments
\IfFileExists{upquote.sty}{\usepackage{upquote}}{}
\IfFileExists{microtype.sty}{% use microtype if available
  \usepackage[]{microtype}
  \UseMicrotypeSet[protrusion]{basicmath} % disable protrusion for tt fonts
}{}
\makeatletter
\@ifundefined{KOMAClassName}{% if non-KOMA class
  \IfFileExists{parskip.sty}{%
    \usepackage{parskip}
  }{% else
    \setlength{\parindent}{0pt}
    \setlength{\parskip}{6pt plus 2pt minus 1pt}}
}{% if KOMA class
  \KOMAoptions{parskip=half}}
\makeatother
\usepackage{xcolor}
\usepackage[margin=1in]{geometry}
\usepackage{color}
\usepackage{fancyvrb}
\newcommand{\VerbBar}{|}
\newcommand{\VERB}{\Verb[commandchars=\\\{\}]}
\DefineVerbatimEnvironment{Highlighting}{Verbatim}{commandchars=\\\{\}}
% Add ',fontsize=\small' for more characters per line
\usepackage{framed}
\definecolor{shadecolor}{RGB}{248,248,248}
\newenvironment{Shaded}{\begin{snugshade}}{\end{snugshade}}
\newcommand{\AlertTok}[1]{\textcolor[rgb]{0.94,0.16,0.16}{#1}}
\newcommand{\AnnotationTok}[1]{\textcolor[rgb]{0.56,0.35,0.01}{\textbf{\textit{#1}}}}
\newcommand{\AttributeTok}[1]{\textcolor[rgb]{0.77,0.63,0.00}{#1}}
\newcommand{\BaseNTok}[1]{\textcolor[rgb]{0.00,0.00,0.81}{#1}}
\newcommand{\BuiltInTok}[1]{#1}
\newcommand{\CharTok}[1]{\textcolor[rgb]{0.31,0.60,0.02}{#1}}
\newcommand{\CommentTok}[1]{\textcolor[rgb]{0.56,0.35,0.01}{\textit{#1}}}
\newcommand{\CommentVarTok}[1]{\textcolor[rgb]{0.56,0.35,0.01}{\textbf{\textit{#1}}}}
\newcommand{\ConstantTok}[1]{\textcolor[rgb]{0.00,0.00,0.00}{#1}}
\newcommand{\ControlFlowTok}[1]{\textcolor[rgb]{0.13,0.29,0.53}{\textbf{#1}}}
\newcommand{\DataTypeTok}[1]{\textcolor[rgb]{0.13,0.29,0.53}{#1}}
\newcommand{\DecValTok}[1]{\textcolor[rgb]{0.00,0.00,0.81}{#1}}
\newcommand{\DocumentationTok}[1]{\textcolor[rgb]{0.56,0.35,0.01}{\textbf{\textit{#1}}}}
\newcommand{\ErrorTok}[1]{\textcolor[rgb]{0.64,0.00,0.00}{\textbf{#1}}}
\newcommand{\ExtensionTok}[1]{#1}
\newcommand{\FloatTok}[1]{\textcolor[rgb]{0.00,0.00,0.81}{#1}}
\newcommand{\FunctionTok}[1]{\textcolor[rgb]{0.00,0.00,0.00}{#1}}
\newcommand{\ImportTok}[1]{#1}
\newcommand{\InformationTok}[1]{\textcolor[rgb]{0.56,0.35,0.01}{\textbf{\textit{#1}}}}
\newcommand{\KeywordTok}[1]{\textcolor[rgb]{0.13,0.29,0.53}{\textbf{#1}}}
\newcommand{\NormalTok}[1]{#1}
\newcommand{\OperatorTok}[1]{\textcolor[rgb]{0.81,0.36,0.00}{\textbf{#1}}}
\newcommand{\OtherTok}[1]{\textcolor[rgb]{0.56,0.35,0.01}{#1}}
\newcommand{\PreprocessorTok}[1]{\textcolor[rgb]{0.56,0.35,0.01}{\textit{#1}}}
\newcommand{\RegionMarkerTok}[1]{#1}
\newcommand{\SpecialCharTok}[1]{\textcolor[rgb]{0.00,0.00,0.00}{#1}}
\newcommand{\SpecialStringTok}[1]{\textcolor[rgb]{0.31,0.60,0.02}{#1}}
\newcommand{\StringTok}[1]{\textcolor[rgb]{0.31,0.60,0.02}{#1}}
\newcommand{\VariableTok}[1]{\textcolor[rgb]{0.00,0.00,0.00}{#1}}
\newcommand{\VerbatimStringTok}[1]{\textcolor[rgb]{0.31,0.60,0.02}{#1}}
\newcommand{\WarningTok}[1]{\textcolor[rgb]{0.56,0.35,0.01}{\textbf{\textit{#1}}}}
\usepackage{graphicx}
\makeatletter
\def\maxwidth{\ifdim\Gin@nat@width>\linewidth\linewidth\else\Gin@nat@width\fi}
\def\maxheight{\ifdim\Gin@nat@height>\textheight\textheight\else\Gin@nat@height\fi}
\makeatother
% Scale images if necessary, so that they will not overflow the page
% margins by default, and it is still possible to overwrite the defaults
% using explicit options in \includegraphics[width, height, ...]{}
\setkeys{Gin}{width=\maxwidth,height=\maxheight,keepaspectratio}
% Set default figure placement to htbp
\makeatletter
\def\fps@figure{htbp}
\makeatother
\setlength{\emergencystretch}{3em} % prevent overfull lines
\providecommand{\tightlist}{%
  \setlength{\itemsep}{0pt}\setlength{\parskip}{0pt}}
\setcounter{secnumdepth}{-\maxdimen} % remove section numbering
\ifLuaTeX
  \usepackage{selnolig}  % disable illegal ligatures
\fi
\IfFileExists{bookmark.sty}{\usepackage{bookmark}}{\usepackage{hyperref}}
\IfFileExists{xurl.sty}{\usepackage{xurl}}{} % add URL line breaks if available
\urlstyle{same} % disable monospaced font for URLs
\hypersetup{
  pdftitle={Bio Project},
  hidelinks,
  pdfcreator={LaTeX via pandoc}}

\title{Bio Project}
\author{}
\date{\vspace{-2.5em}2022-12-15}

\begin{document}
\maketitle

\hypertarget{zahra-soukhtedel---98105138}{%
\subparagraph{Zahra Soukhtedel -
98105138}\label{zahra-soukhtedel---98105138}}

\hypertarget{bahar-oveis-gharan---98106242}{%
\subparagraph{Bahar Oveis Gharan -
98106242}\label{bahar-oveis-gharan---98106242}}

\hypertarget{lachin-naghashyar---98110179}{%
\subparagraph{Lachin Naghashyar -
98110179}\label{lachin-naghashyar---98110179}}

\begin{Shaded}
\begin{Highlighting}[]
\FunctionTok{setwd}\NormalTok{(}\StringTok{"\textasciitilde{}/Desktop/bio\_proj/"}\NormalTok{)}
\FunctionTok{library}\NormalTok{(GEOquery)}
\FunctionTok{library}\NormalTok{(limma)}
\FunctionTok{library}\NormalTok{(umap)}
\FunctionTok{library}\NormalTok{(pheatmap)}
\FunctionTok{library}\NormalTok{(gplots)}
\FunctionTok{library}\NormalTok{(ggplot2)}
\FunctionTok{library}\NormalTok{(reshape2)}
\FunctionTok{library}\NormalTok{(plyr)}
\end{Highlighting}
\end{Shaded}

\hypertarget{q1}{%
\subsection{Q1}\label{q1}}

\hypertarget{q2}{%
\subsection{Q2}\label{q2}}

If we look at the distributions illustrated in the box plots, we can see
that their distributions are close to each other and their boxed have a
large overlap and the the medians are similar to each other. Hence, we
don't need to normalize the data. Moreover, the values are under 20
which means they already have logarithmic scale.

\begin{figure}
\includegraphics[width=1\linewidth]{Data/dists_boxplot} \caption{A caption}\label{fig:pressure}
\end{figure}

We can also obtain the results.txt the same way as the videos from the
GEO website.

\begin{Shaded}
\begin{Highlighting}[]
\NormalTok{res }\OtherTok{\textless{}{-}} \FunctionTok{read.delim}\NormalTok{(}\StringTok{"Results/results.txt"}\NormalTok{)}
\FunctionTok{head}\NormalTok{(res)}
\end{Highlighting}
\end{Shaded}

\begin{verbatim}
##        ID adj.P.Val  P.Value         t        B     logFC Gene.symbol
## 1 8016932  3.62e-19 1.12e-23 -15.30903 43.23352 -5.563501         MPO
## 2 7970737  4.84e-19 2.99e-23 -15.02516 42.28692 -5.250065        FLT3
## 3 7989647  6.31e-19 5.86e-23 -14.83304 41.64005 -4.559135    KIAA0101
## 4 7982663  1.66e-18 2.06e-22 -14.47666 40.42674 -2.756554       BUB1B
## 5 8083422  1.94e-18 3.00e-22 -14.37104 40.06382 -2.996816      SUCNR1
## 6 7926259  3.71e-18 6.89e-22 -14.13865 39.25991 -2.318848       MCM10
##                                                    Gene.title
## 1                                             myeloperoxidase
## 2                               fms related tyrosine kinase 3
## 3                                                    KIAA0101
## 4           BUB1 mitotic checkpoint serine/threonine kinase B
## 5                                        succinate receptor 1
## 6 minichromosome maintenance 10 replication initiation factor
\end{verbatim}

Use the R script generated in the website for the specific data that's
been chosen.

\begin{verbatim}
## Found 1 file(s)
\end{verbatim}

\begin{verbatim}
## GSE48558_series_matrix.txt.gz
\end{verbatim}

\begin{verbatim}
## Using locally cached version: Data//GSE48558_series_matrix.txt.gz
\end{verbatim}

\begin{verbatim}
## Using locally cached version of GPL6244 found here:
## Data//GPL6244.annot.gz
\end{verbatim}

\hypertarget{quality-control}{%
\subsubsection{Quality Control}\label{quality-control}}

In each step we might have some errors, hence we should always check the
quality of data before proceeding to next steps.

\hypertarget{normalization}{%
\paragraph{Normalization}\label{normalization}}

Raw gene expression data from the high-throughput technologies must be
normalized to remove technical variation so that meaningful biological
comparisons can be made. Data normalization is a crucial step in the
gene expression analysis as it ensures the validity of its downstream
analyses. \textbackslash{} First we check if data is normalized or not

\begin{verbatim}
## pdf 
##   2
\end{verbatim}

It seems that the data is normalized, however, we can also perform
normalization on it using the code below, which is not necessary.

\begin{Shaded}
\begin{Highlighting}[]
\NormalTok{ex }\OtherTok{\textless{}{-}}\FunctionTok{normalizeQuantiles}\NormalTok{(ex) }
\FunctionTok{exprs}\NormalTok{(gset) }\OtherTok{\textless{}{-}}\NormalTok{ ex}
\end{Highlighting}
\end{Shaded}

\hypertarget{correlation-heatmap}{%
\paragraph{Correlation Heatmap}\label{correlation-heatmap}}

Correlation heatmaps can be used to find potential relationships between
variables and to understand the strength of these relationships. In
addition, correlation plots can be used to identify outliers and to
detect linear and nonlinear relationships. The heatmap may also be
combined with heirarchical clustering methods on rows and columns which
group genes and/or samples together based on the similarity of their
gene expression pattern in a heirarchical way.

\begin{verbatim}
## pdf 
##   3
\end{verbatim}

\hypertarget{pca}{%
\paragraph{PCA}\label{pca}}

\begin{verbatim}
## pdf 
##   2
\end{verbatim}

As expected, most of the variace is along the first principal component.
We can check the names and the dimension of the PC.

\begin{Shaded}
\begin{Highlighting}[]
\FunctionTok{names}\NormalTok{(pc)}
\end{Highlighting}
\end{Shaded}

\begin{verbatim}
## [1] "sdev"     "rotation" "center"   "scale"    "x"
\end{verbatim}

\begin{Shaded}
\begin{Highlighting}[]
\FunctionTok{dim}\NormalTok{(pc}\SpecialCharTok{$}\NormalTok{x)}
\end{Highlighting}
\end{Shaded}

\begin{verbatim}
## [1] 32321   170
\end{verbatim}

\begin{Shaded}
\begin{Highlighting}[]
\CommentTok{\# colnames(pc$x)}
\end{Highlighting}
\end{Shaded}

We can also scale the PCs by subtracting the mean of each component.PCA
will be extremely biased towards the first feature being the first
principle component, regardless of the actual maximum variance within
the data. This is why it's so important to standardize the values first.

\begin{Shaded}
\begin{Highlighting}[]
\NormalTok{ex.scale }\OtherTok{\textless{}{-}} \FunctionTok{t}\NormalTok{(}\FunctionTok{scale}\NormalTok{(}\FunctionTok{t}\NormalTok{(ex), }\AttributeTok{scale =} \ConstantTok{FALSE}\NormalTok{)) }\CommentTok{\#we don\textquotesingle{}t want to eliminate variance}
\CommentTok{\# mean(ex.scale[,1])}
\NormalTok{pc }\OtherTok{\textless{}{-}} \FunctionTok{prcomp}\NormalTok{(ex.scale)}
\FunctionTok{pdf}\NormalTok{(}\StringTok{"Results/PC\_scaled.pdf"}\NormalTok{)}
\FunctionTok{plot}\NormalTok{(pc)}
\FunctionTok{plot}\NormalTok{(pc}\SpecialCharTok{$}\NormalTok{x[,}\DecValTok{1}\SpecialCharTok{:}\DecValTok{2}\NormalTok{])}
\FunctionTok{dev.off}\NormalTok{()}
\end{Highlighting}
\end{Shaded}

\begin{verbatim}
## pdf 
##   2
\end{verbatim}

\begin{Shaded}
\begin{Highlighting}[]
\NormalTok{pcr }\OtherTok{\textless{}{-}} \FunctionTok{data.frame}\NormalTok{(pc}\SpecialCharTok{$}\NormalTok{r[,}\DecValTok{1}\SpecialCharTok{:}\DecValTok{3}\NormalTok{], }\AttributeTok{Group=}\NormalTok{gr)}
\FunctionTok{head}\NormalTok{(pcr)}
\end{Highlighting}
\end{Shaded}

\begin{verbatim}
##                     PC1        PC2           PC3 Group
## GSM1180750 -0.031565140 0.07162482  0.0144240962   AML
## GSM1180751 -0.055489995 0.09074739  0.0014317626   AML
## GSM1180752 -0.017952039 0.02903359 -0.0024554711   AML
## GSM1180753 -0.071985213 0.12583838  0.0008634052   AML
## GSM1180754  0.006820641 0.03727649 -0.0019770972   AML
## GSM1180755 -0.051211058 0.06196639 -0.0144744377   AML
\end{verbatim}

\begin{Shaded}
\begin{Highlighting}[]
\FunctionTok{pdf}\NormalTok{(}\StringTok{"Results/PCA\_samples.pdf"}\NormalTok{)}
\FunctionTok{ggplot}\NormalTok{(pcr, }\FunctionTok{aes}\NormalTok{(PC1, PC2, }\AttributeTok{color=}\NormalTok{Group)) }\SpecialCharTok{+} \FunctionTok{geom\_point}\NormalTok{(}\AttributeTok{size=}\DecValTok{3}\NormalTok{) }\SpecialCharTok{+} \FunctionTok{geom\_point}\NormalTok{(}\AttributeTok{size=}\DecValTok{3}\NormalTok{) }\SpecialCharTok{+} \FunctionTok{theme\_bw}\NormalTok{()}
\FunctionTok{dev.off}\NormalTok{()}
\end{Highlighting}
\end{Shaded}

\begin{verbatim}
## pdf 
##   2
\end{verbatim}

\end{document}
